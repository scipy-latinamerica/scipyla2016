\documentclass[a4paper,12pt]{article}
\usepackage[portuguese]{babel}
\usepackage[utf8]{inputenc}

\begin{document}
\thispagestyle{empty}
% FIXME Corrigir destinatário.
Prezado \ldots

vemos por meio dessa carta avisar da SciPy Latin America 2016, a quarta conferência anual de Computação Científica com Python, será realizada nos dias 25, 26 e 27 de maio de 2016, em Florianópolis / Brasil.
A conferência SciPy Latin America é focada em aplicações científicas e afins que
utilizam Python, seguindo o exemplo de outras conferências regionais de mesmo tema que ocorrem na Europa (EuroSciPy), Índia (SciPy India) e Estados Unidos (SciPy). Esse encontro sucede 3 outros encontros de mesma temática ocorridos na Argentina em 2013, 2014 e 2015.

O citado encontro consistirá de várias oficinas e palestras voltadas a
pesquisadores, professores de diferentes níveis educacionais, estudantes,
profissionais e empresários interessados em
computação científica com o uso de softwares \emph{open source} como Python para, mas não limitado a, Ciências Exatas, Biológicas, Humanas, e da Terra.

Tomando-se por base os encontros anteriores realizados na Argentina e Brasil, espera-se uma participação de aproximadamente 200 participantes provenientes das mais diferentes regiões.
A organização da conferência oferece auxílio financeiro parcial para participantes,
palestrantes e membros do comitê organizador de forma que o evento seja inclusivo.
A verba para o auxílio financeiro, normalmente, é proveniente
de instituições públicas e privadas.

% FIXME Adaptar para cada caso.
Gostaríamos do seu apoio para a realização do SciPy Latin America 2016.
Estamos procurando \ldots

Para maiores informações, entre em contato com um dos organizadores do SciPy
Latin America 2016: Ivan Ogasawara, \texttt{ivan.ogasawara@gmail.com} ou (48)
9909\textendash 0207, ou Letícia Portella, \texttt{leportella@gmail.com}.

\vspace{1cm}

Obrigado pela atenção,

\vspace{0.5cm}

Comissão Organizadora do SciPyLA2016

\end{document}
